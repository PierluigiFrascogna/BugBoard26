\documentclass[a4paper, 11pt]{article}
\title{BugBoard 2026}
\author{Marco Festa, Alessandro Giglio, Pierluigi Frascogna}

\usepackage[a4paper, margin=2.5cm, headheight=39pt]{geometry}
\usepackage[italian]{babel}
\usepackage{graphicx}
\usepackage{svg}
\usepackage{enumitem}
\usepackage[table]{xcolor}
\usepackage{tabularx}
\usepackage{multirow}
\usepackage{float}
\usepackage{fancyhdr}
\usepackage{array}
\usepackage{hyperref}
\usepackage[nonumberlist]{glossaries}

\makeglossaries
\newglossaryentry{issue}{
    name=issue,
    description={il "problema" identificato all'interno di un progetto che concerne l'utente}
}

\newglossaryentry{sistema}{
    name=sistema,
    description={BugBoard26}
}

\newglossaryentry{piattaforma}{
    name=piattaforma,
    description={Vedi sistema},
    see=sistema
}

\renewcommand{\arraystretch}{1.4}
\graphicspath{{images/}}
\svgpath{{images/}}
\pagestyle{fancy}
\fancyhf{}  % cancella header/footer di default di Latex
\renewcommand{\headrulewidth}{0.1pt}
\fancyhead[L]{{\includesvg[height=1.2cm]{BugBoard26_Logo}}} 
\fancyhead[R]{Pag. \thepage} 


\begin{document}

\begin{titlepage}
    \centering
    {\LARGE UNIVERSITÀ DEGLI STUDI FEDERICO II} \\[1em]
    {\large SCUOLA POLITECNICA E DELLE SCIENZE DI BASE} \\[0.5em]
    {\small DIPARTIMENTO DI INGEGNERIA ELETTRICA E TECNOLOGIE DELL'INFORMAZIONE} \\[2em]

    \includesvg[width=0.50\linewidth]{Federico_II_Logo} \\[2em]

    {\large CORSO DI LAUREA IN INFORMATICA} \\[0.5em]
    {\large INSEGNAMENTO DI INGEGNERIA DEL SOFTWARE} \\[2em]

    {\large ANNO ACCADEMICO 2025/2026} \\[2em]

    {\LARGE \textbf{Progettazione e sviluppo della piattaforma BugBoard26}}
\end{titlepage}

\newpage
\tableofcontents  % Indice

\newpage
\section{INTRODUZIONE}

\subsection{Chi siamo}
Benvenuto su \textbf{BugBoard26}! \\
BugBoard26 è una \gls{piattaforma} di \textit{issues handling} che fornisce una soluzione unica per:
\begin{itemize}
    \item Dividere in modo facile developer in progetti.
    \item Segnalare e gestire intuitivamente \gls{issue} di vario tipo.
    \item Gestire in modo efficiente tutte le persone coinvolte in un progetto (anche non sviluppatori) tramite una gerarchia di utenze.
\end{itemize}

\printglossaries % Glossario

\newpage
\section{INGEGNERIA DEI REQUISITI}

\subsection{Casi d'uso}
In questa sezione ci interesseremo all'individuazione dei casi d'uso. Come si può evincere dallo use case diagram riportato qui di seguito:

\begin{figure}[H]
    \centering
    \includesvg[width=0.5\linewidth]{UseCase_Diagram}
    \caption{Use Case Diagram}
    \label{fig:use_case_diagram}
\end{figure}

Tutti gli utilizzatori della \gls{piattaforma} possono essere divisi in tre grandi macrocategorie:
\begin{itemize}
    \item \textbf{Viewer}: individuati anche nella figura di uno stakeholder. Sono utilizzatori, non necessariamente del settore, che hanno comunque interesse a visualizzare le \gls{issue} legate al progetto senza poterle però aggiungere o modificare.
    \item \textbf{Developer}: rappresentano la stragrande maggioranza di utilizzatori della \gls{piattaforma}. I developer sono coloro che contribuiscono attivamente all'individuazione e risoluzione delle \gls{issue}.
    \item \textbf{Admin}: rappresentano l'estensione di un developer con permessi di creazione e gestione di altre utenze.
\end{itemize}

\newpage
\subsection{Individuazione delle personas}
Ora esamineremo alcune personas che rispecchiano alcuni dei tipi di utilizzatori della nostra \gls{piattaforma}.

% Prima tabella personas
\begin{table}[H]
    \centering
    \begin{tabular}{|p{0.9\linewidth}|}
        \hline
        Nome: Mark Party                                                                                                                                                                                                                                                            \\
        Età: 52 anni                                                                                                                                                                                                                                                                \\
        Posizione: Product Owner                                                                                                                                                                                                                                                    \\
        \hline
        \begin{minipage}[t]{\linewidth}
            \textbf{Obiettivi:}
            \begin{itemize}
                \item Sarebbe molto utile poter vedere l'andazzo del team così da sapere in che direzione indirizzarlo e come gestirlo.
            \end{itemize}
        \end{minipage}                                                                                                                                                                                                                                              \\
        \hline
        \textbf{Bio:}                                                                                                                                                                                                                                                               \\
        Sono un economista italo-americano di Boston. Nell'arco della mia carriera mi sono ritrovato a gestire diverse start-up e gruppi di lavoro, nonostante non capisca molto di queste diavolerie informatiche, mi ritengo molto più capace a gestire e portare avanti prodotti \\
        \hline
    \end{tabular}
\end{table}

% Seconda tabella personas
\begin{table}[H]
    \centering
    \begin{tabular}{|p{0.9\linewidth}|}
        \hline
        Nome: Aleksander Lilia                                                                                                                                                                                                                          \\
        Età: 24 anni                                                                                                                                                                                                                                    \\
        Posizione: Developer                                                                                                                                                                                                                            \\
        \hline
        \begin{minipage}[t]{\linewidth}
            \textbf{Obiettivi:}
            \begin{itemize}
                \item Per lavorare in modo efficiente devo sapere quali problemi devo sistemare e magari avere del feedback dai miei colleghi.
                \item Nel caso dovessi trovare dei problemi, vorrei avere un modo comodo per segnalarli in modo dettagliato.
                \item Una volta risolti tali problemi vorrei poter segnalarlo al mio team.
            \end{itemize}
        \end{minipage}                                                                                                                                                                                                                  \\
        \hline
        \textbf{Bio:}                                                                                                                                                                                                                                   \\
        Sono un developer di Izdebki, dopo essermi laureato all'università di Cracovia mi sono trasferito a Napoli per lavoro e per amore. Sono grande amatore della filosofia “work smarter not harder” che cerco di applicare in ogni modo possibile. \\
        \hline
    \end{tabular}
\end{table}

% Terza tabella personas
\begin{table}[H]
    \centering
    \begin{tabular}{|p{0.9\linewidth}|}
        \hline
        Nome: Pierrelouis Frascout                                                                                                                                                                                                                      \\
        Età: 37 anni                                                                                                                                                                                                                                    \\
        Posizione: Team leader                                                                                                                                                                                                                          \\
        \hline
        \begin{minipage}[t]{\linewidth}
            \textbf{Obiettivi:}
            \begin{itemize}
                \item Voglio poter gestire i membri del mio team in modo chiaro ed efficiente.
                \item Voglio poter tenere traccia dei progressi fatti dal mio team e come si sta comportando.
                \item Voglio condividere con tutte le persone interessate, l'andamento del nostro team.
            \end{itemize}
        \end{minipage}                                                                                                                                                                                                                  \\
        \hline
        \textbf{Bio:}
        Sono un software engineer di Nantes ma ho vissuto buona parte della mia vita a Roma. Sono una persona risolutiva ed estremamente orientata al pratico e questo si riflette nel mio modo di lavorare                                                                                                                                                                                                                                   \\
        \hline
    \end{tabular}
\end{table}


\subsection{Requisiti non funzionali e di dominio}
\subsubsection{Requisiti non funzionali}
I requisiti non funzionali da noi individuati sono:
\begin{itemize}
    \item \textbf{Permanenza dei dati:} attraverso un database non MBaaS.
    \item \textbf{Utilizzo di linguaggi orientati agli oggetti.}
    \item \textbf{Implementazione di un modello Client-Server.}
    \item \textbf{Elevata manutenibilità.}
    \item \textbf{Efficienza e affidabilità:} non essendo la \gls{piattaforma} safety-critical, limitazioni di tempo e memoria occupata sono da considerarsi standard e ragionevoli.
\end{itemize}

\subsubsection{Requisiti di dominio}
Non sono stati individuati requisiti di dominio particolarmente differenti da quelli forniti nella traccia.

\newpage
\subsection{Formalizzazione di un requisito}
Qui di seguito riportiamo la formalizzazione di un requisito quale la visualizzazione di una \gls{issue}, prima mediante il suo mockup:

\begin{figure}[H]
    \centering
    \includesvg[width=0.9\linewidth]{Visualizza_Issue_Mockup}
    \caption{Visualizza \gls{issue}}
    \label{fig:visualizza_issue_mockup}
\end{figure}

\newpage
E qui di seguito riportiamo l'inerente tabella di Cockburn:


% --- definizione colori ---
\definecolor{violet1}{HTML}{D87CC0}
\definecolor{violet2}{HTML}{E9A2D0}
\definecolor{orange1}{HTML}{F8B88B}
\definecolor{orange2}{HTML}{FCD4B2}
\definecolor{yellow1}{HTML}{FFC75F}
\definecolor{yellow2}{HTML}{FFF8C6}


\begin{table}[H]
    \centering
    % Colonna sinistra: 'm' per allineamento verticale centrato
    % Colonna destra: 'p' per il testo a capo
    \begin{tabular}{|m{0.3\linewidth}|p{0.66\linewidth}|}
        \hline
        \rowcolor{violet1}
        \textbf{USE CASE}                               & \textit{Visualizza \gls{issue}}                                              \\
        \hline
        \rowcolor{violet2}
        \textbf{Goal}                                   & Un utente vuole visualizzare una \gls{issue}, le sue proprietà e i commenti. \\
        \hline
        \rowcolor{violet2}
        \textbf{Preconditions}                          & L'utente ha un account e si è autenticato.                             \\
        \hline
        \rowcolor{violet2}
        \textbf{Success end conditions}                 & Il \gls{sistema} mostra la \gls{issue} e i suoi commenti.                          \\
        \hline
        \rowcolor{violet2}
        \textbf{Failed end conditions}                  & Il \gls{sistema} mostra una pagina di errore.                                \\
        \hline
        \rowcolor{violet2}
        \textbf{Primary actor}                          & Qualsiasi tipo di user.                                                \\
        \hline
        \rowcolor{violet2}
        \textbf{Trigger}                                & L'utente fa accesso.                                                   \\
        \hline

        % --- Main scenario (tabella annidata) ---
        \cellcolor{orange1}
        \textbf{Main scenario}                          &
        % Questa tabularx usa \linewidth per riempire la cella P genitore
        \begin{tabularx}{\linewidth}{|X|X|X|}
            \hline
            \rowcolor{orange1}
            \textbf{Step n.} & \textbf{Utente}       & \textbf{Sistema} \\
            \hline
            \rowcolor{orange2}
            1                &                       & Mostra M1        \\
            \hline
            \rowcolor{orange2}
            2                & Clicca su un progetto &                  \\
            \hline
            \rowcolor{orange2}
            3                &                       & Mostra M2        \\
            \hline
            \rowcolor{orange2}
            4                & Clicca su una \gls{issue}   &                  \\
            \hline
            \rowcolor{orange2}
            5                &                       & Mostra M3        \\
            \hline
        \end{tabularx}                                                              \\
        \hline

        % --- Extensions (tabella annidata) ---
        % L'allineamento 'm' della cella sinistra centra questo testo
        % verticalmente rispetto alla tabella a destra.
        \cellcolor{yellow1}
        \textbf{Extension n° 1 (User has no projects)}  &
        \begin{tabularx}{\linewidth}{|X|X|X|}
            \hline % Inizia subito con \hline
            \rowcolor{yellow2}
            1a &  & Mostra M1a \\
            \hline
        \end{tabularx}                                                                                  \\
        \hline

        \cellcolor{yellow1}
        \textbf{Extension n° 2 (Project has no issues)} &
        \begin{tabularx}{\linewidth}{|X|X|X|}
            \hline
            \rowcolor{yellow2}
            2b &  & Mostra M2b \\
            \hline
        \end{tabularx}                                                                                     \\
        \hline

        \cellcolor{yellow1}
        \textbf{Extension n° 3 (Generic error)}         &
        \begin{tabularx}{\linewidth}{|X|X|X|}
            \hline
            \rowcolor{yellow2}
            1,3,5 err &  & Mostra MErr \\
            \hline
        \end{tabularx}                                                                                     \\

        % L'ultimo \hline chiude la tabella
        \hline
    \end{tabular}
\end{table}


\newpage
\section{SYSTEM DESIGN}

\subsection{Architettura del sistema}
L'architettura del sistema adottata per la realizzazione della \gls{piattaforma} è di tipo \textbf{Client-Server}, più specificatamente a microservizi. Questa scelta, ideale per applicazioni web, disaccoppia nettamente il Frontend (lato utente) dal Backend (logica e dati su server).
\\ I microservizi da noi individuati sono:
\begin{itemize}
    \item \textbf{Servizio di gestione utenze:} si occupa della gestione delle utenze e dell'autenticazione.
    \item \textbf{Servizio di gestione issue:} si occupa della gestione delle \gls{issue} e dei relativi commenti.
\end{itemize}
L'organizzazione del Backend in microservizi distinti è stata guidata da tre obiettivi principali:
\begin{itemize}
    \item \textbf{Scalabilità Indipendente:} Consente di allocare risorse computazionali in modo mirato. È possibile, ad esempio, scalare orizzontalmente il servizio di gestione delle issue (soggetto a traffico più intenso) senza dover replicare inutilmente il servizio di gestione utenze.
    \item \textbf{Disaccoppiamento e Manutenibilità:} La suddivisione in moduli riduce la complessità del codice del singolo servizio. Questo favorisce uno sviluppo distribuibile, manutenibile, evolvibile e con test più mirati.
    \item \textbf{Tolleranza ai guasti:} Un eventuale errore critico in un microservizio non compromette necessariamente la disponibilità dell'intera piattaforma.
\end{itemize}


\subsection{Decomposizione del sistema}
Il sistema è composto dai seguenti elementi:
\begin{enumerate}
        \item \textbf{Frontend:} interfaccia utente accessibile tramite browser web.
        \begin{itemize}[label=-]
            \item Login/Registrazione
            \item Visualizzazione progetti e issue
            \item Creazione/modifica progetti e issue
            \item Gestione profili utente (solo per admin)
        \end{itemize}
        
        \item \textbf{Backend:} microservizi che gestiscono le funzionalità principali della \gls{piattaforma}.
         \begin{itemize}[label=-]
            \item \textbf{Utenze:} gestione utenti e profili
            \item \textbf{Gestione issue:} gestione progetti e issue
        \end{itemize}

        \item \textbf{Database:} sistema di gestione dei dati persistenti.
        \begin{itemize}[label=-]
            \item \textbf{Utenze:} memorizzazione dati utenti e profili
            \item \textbf{Gestione issue:} memorizzazione dati progetti e issue
        \end{itemize}
\end{enumerate}

% Inserire immagine che rappresenta l'architettura

\subsection{Scelta delle tecnologie}
Per la realizzazione della \gls{piattaforma} BugBoard26, abbiamo scelto le seguenti tecnologie:
\begin{itemize}
    \item \textbf{Frontend:} \textit{Angular.ts} per la costruzione dell'interfaccia utente, grazie alla sua modularità e facilità di integrazione con backend RESTful.
    \item \textbf{Backend:} \textit{Java Spring} per la creazione dei microservizi, grazie alla sua robustezza.
    \item \textbf{Database:} \textit{PostgreSQL} per la gestione dei dati relazionali, grazie alla sua affidabilità e scalabilità.
\end{itemize}
\newpage

\section{SOFTWARE DESIGN}
\subsection{Persistenza dei dati}
Per la parte di persistenza di dati ci siamo affidati a Supabase, non per le funzionalità di MBaaS, ma come provider di database.
Come annunciato in precedenza, il backend è diviso in due microservizi, ognuno dei quali fornito del proprio database. Qui di seguito
riportiamo lo schema per entrambi:

\includegraphics[width=1\linewidth]{usersDB}
Questo è il database dello UsersService. La sua funzione principale è quella di contenere per intero i dati delle varie utenze. Separare
le informazioni delle utenze dalle informazioni proprie e caratteristiche di BugBoard26 ci permette una certa scalabilità oltre a impostare
BugBoard26 verso quella che potrebbe diventare una famiglia di servizi in un futuro ipotetico.

\includegraphics[width=1\linewidth]{bugBoardDB}
Questo è il database del BugBoardService, cuore pulsante della \gls{piattaforma}. È qui che sono conservate tutte le informazioni riguardanti
progetti, \gls{issue} e interazioni degli utenti con queste. Questo possiede un sottoinsieme delle informazioni degli utenti, al fine di semplificare
le operazioni intra-database e per permettere il suo corretto funzionamento anche nel caso di down del database dello UsersService.

\subsection{Strumenti di versioning}
Lo strumento di versioning da noi utilizzato è stato GitHub, scelta facile e dettata dalla sua diffusione e semplicità d'uso. È possibile 
visualizzare la repo di BugBoard26 al seguente link: \\
\url{https://github.com/PierluigiFrascogna/BugBoard26} \\
Ovviamente, qui è anche possibile visualizzare ogni possibile statistica relativa a software e contributors, in modo molto più esaustivo di 
quanto potremmo mai offrire noi.

\subsection{Qualità del codice}
Per la generazione di report sulla qualità del codice la nostra scelta è ricaduta su SonarQube. Questa scelta è stata dettata dalla sua
estensività, semplicità d'uso e la profonda integrazione con GitHub. Qui di seguito riportiamo il report sulla qualità del codice
inerente al backend: 
\includegraphics[width=1\linewidth]{sonarqubeReport}
\newpage

\section{TESTING}
\subsection{Test Plan}
La funzionalità che abbiamo deciso di testare è quella inerente alle \gls{issue} in quanto considerata da noi fondamentale 
per la \gls{piattaforma} stessa. Nello specifico testeremo i metodi
"IssueResponse postNewIssue(UUID uuid\_project, IssueRequest issueRequest)" della classe IssueAPI e 
"List<IssueResponse> getIssuesByProject(UUID uuid\_project, string type, string priority, string state)" della classe IssueService. 
La metodologia utilizzata per testare entrambi i metodi è quella black-box, nello specifico R-WECT, in quanto giusto compromesso tra mole
di test ed esaustività degli stessi. Ora illustremo le informazioni relative al testing di ognuno dei metodi.

\subsubsection{\texttt{IssueResponse postNewIssue(UUID uuid\_project, IssueRequest issueRequest)}}
Questo metodo, come si può intuire dal nome, si occupa di creare nuove issue e racchiuderele nel corretto formato JSON da spedire al 
frontend. Al fine del testing sono state individuate le seguenti classi di equivalenza (al fine di una maggiore chiarezza, le classi di 
equivalenza per parametri diversi di uno stesso metodo saranno in ordine crescente):


 \begin{table}[H]
    \centering
    \caption{Classi di equivalenza per il parametro UUID uuid\_project}
    \begin{tabularx}{\linewidth}{|X|X|X|}
        \hline
        \textbf{Classi di equivalenza}  & \textbf{Descrizione}                                                                      &\textbf{Valido}    \\
        \hline
        CE1                             & uuid valido, ossia conforme allo standard UUID e presente all'interno del database        & SI                \\
        \hline
        CE2                             & uuid non valido, ossia non conforme allo standard UUID                                    & NO                \\
        \hline
        CE3                             & uuid valido, ossia conforme allo standard UUID ma non presente all'interno del database   & NO                \\
        \hline
    \end{tabularx}
\end{table}

\begin{table}[H]
    \centering
    \caption{Classi di equivalenza per il parametro issueRequest}
    \begin{tabularx}{\linewidth}{|X|X|X|}
        \hline
        \textbf{Classi di equivalenza}  & \textbf{Descrizione}                                                                              &\textbf{Valido}    \\
        \hline
        CE4                             & un oggetto di tipo issueRequest valido, ossia correttamente allocabile a partire da un JSON       & SI                \\
        \hline
        CE5                             & un oggetto di tipo issueRequest non valido, ossia impossibile da allocare a partire da un JSON    & NO                \\
        \hline
    \end{tabularx}
\end{table}

\begin{table}[H]
    \centering
    \caption{Oracolo per test del metodo IssueResponse postNewIssue(UUID uuid\_project, IssueRequest issueRequest)}
    \begin{tabularx}{\linewidth}{|X|X|X|}
        \hline
        \textbf{Caso di test} & \textbf{Classe di equivalenza} & \textbf{Esito atteso}                  \\
        \hline
        TC1                   & CE1, CE4                      & Successo: IssueResponse corretta, con tutti i dati della issue creata   \\
        \hline
        TC2                   & CE2, CE4                      & Errore: 400     \\
        \hline
        TC3                   & CE3, CE4                      & Errore: 404     \\
        \hline
        TC4                   & CE1, CE5                      & Errore: 400     \\
        \hline
    \end{tabularx}
\end{table}

\subsubsection{\texttt{List<IssueResponse> getIssuesByProject(UUID uuid\_project, string type, string priority, string state)}}
Questo metodo, come si può intuire dal nome, si occupa di creare nuove istanze della classe Issue.


\begin{table}[H]
    \centering
    \caption{Classi di equivalenza per il parametro uuid\_project}
    \begin{tabularx}{\linewidth}{|X|X|X|}
        \hline
        \textbf{Classi di equivalenza}  & \textbf{Descrizione}                                                                     &\textbf{Valido}    \\
        \hline
        CE1                            & uuid valido, ossia conforme allo standard UUID e presente all'interno del database        & SI                \\
        \hline
        CE2                            & uuid non valido, ossia non conforme allo standard UUID                                    & NO                \\
        \hline
        CE3                            & uuid valido, ossia conforme allo standard UUID ma non presente all'interno del database   & NO                \\
        \hline
    \end{tabularx}
\end{table}

\begin{table}[H]
    \centering
    \caption{Classi di equivalenza per il parametro type}
    \begin{tabularx}{\linewidth}{|X|X|X|}
        \hline
        \textbf{Classi di equivalenza}  & \textbf{Descrizione}                                      &\textbf{Valido}    \\
        \hline
        CE\_BUG                             & stringa "BUG"                                         & SI                \\
        \hline
        CE\_FEATURE                         & stringa "FEATURE"                                     & SI                \\
        \hline
        CE\_DOCUMENTATION                   & stringa "DOCUMENTATION"                               & SI                \\
        \hline
        CE\_QUESTION                        & stringa "QUESTION"                                    & SI                \\
        \hline
        CE\_TOTHERS                         & tutte le altre combinazioni di stringhe possibili     & NO                \\
        \hline
        CE\_TNULL                           & "null"                                                & SI                \\
        \hline
    \end{tabularx}
\end{table}

\begin{table}[H]
    \centering
    \caption{Classi di equivalenza per il parametro priority}
    \begin{tabularx}{\linewidth}{|X|X|X|}
        \hline
        \textbf{Classi di equivalenza}  & \textbf{Descrizione}                                      &\textbf{Valido}    \\
        \hline
        CE\_LOW                             & stringa "LOW"                                         & SI                \\
        \hline
        CE\_MEDIUM                          & stringa "MEDIUM"                                      & SI                \\
        \hline
        CE\_HIGH                            & stringa "HIGH"                                        & SI                \\
        \hline
        CE\_POTHERS                         & tutte le altre combinazioni di stringhe possibili     & NO                \\
        \hline
        CE\_PNULL                           & "null"                                                & SI                \\
        \hline
    \end{tabularx}
\end{table}

\begin{table}[H]
    \centering
    \caption{Classi di equivalenza per il parametro state}
    \begin{tabularx}{\linewidth}{|X|X|X|}
        \hline
        \textbf{Classi di equivalenza}  & \textbf{Descrizione}                                      &\textbf{Valido}    \\
        \hline
        CE\_TODO                            & stringa "TODO"                                        & SI                \\
        \hline
        CE\_PENDING                         & stringa "PENDING"                                     & SI                \\
        \hline
        CE\_DONE                            & stringa "DONE"                                        & SI                \\
        \hline
        CE\_SOTHERS                         & tutte le altre combinazioni di stringhe possibili     & NO                \\
        \hline
        CE\_SNULL                           & "null"                                                & SI                \\
        \hline
    \end{tabularx}
\end{table}

\begin{table}[H]
    \centering
    \caption{Oracolo per test del metodo List<IssueResponse> getIssuesByProject(UUID uuid\_project, string type, string priority, string state)}
    \begin{tabularx}{\linewidth}{|X|X|X|}
        \hline
        \textbf{Caso di test}   & \textbf{Classe di equivalenza}                & \textbf{Esito atteso}                  \\
        \hline
        TC1                     &CE1, CE\_BUG, CE\_LOW, CE\_TODO                & Successo: Lista di IssueResponse corretta, con tutte le issue che rispettano i filtri   \\
        \hline
        TC2                     &CE1, CE\_QUESTION, CE\_MEDIUM, CE\_PENDING     & Successo: Lista di IssueResponse corretta, con tutte le issue che rispettano i filtri   \\
        \hline
        TC3                     &CE1, CE\_FEATURE, CE\_HIGH, CE\_DONE           & Successo: Lista di IssueResponse corretta, con tutte le issue che rispettano i filtri   \\
        \hline
        TC4                     &CE1, CE\_DOCUMENTATION, CE\_PNULL, CE\_SNULL   & Successo: Lista di IssueResponse corretta, con tutte le issue che rispettano i filtri   \\
        \hline
        TC5                     &CE1, CE\_TNULL, CE\_PNULL, CE\_SNULL           & Successo: Lista di IssueResponse corretta, con tutte le issue che rispettano i filtri   \\
        \hline
        TC6                     &CE2, CE\_TNULL, CE\_PNULL, CE\_SNULL           & Errore: 400     \\
        \hline
        TC7                     &CE3, CE\_BUG, CE\_LOW, CE\_PENDING             & Errore: 404     \\
        \hline
        TC8                     &CE1, CE\_TOTHERS, CE\_LOW, CE\_SNULL           & Errore: 400     \\
        \hline
        TC9                     &CE1, CE\_BUG, CE\_POTHERS, CE\_SNULL           & Errore: 400     \\
        \hline
        TC10                    &CE1, CE\_BUG, CE\_LOW, CE\_SOTHERS             & Errore: 400     \\
        \hline
    \end{tabularx}
\end{table}

\end{document}

